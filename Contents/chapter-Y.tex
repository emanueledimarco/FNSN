\chapter{Units}
\label{chap:units}

\section{Microscopic Units for Nuclear, Subnuclear and particle physics}

For all measurements we have seen it appears clearly that the MKSA (kilograms, Kelvin, seconds and Ampers) is not well adapted. It is more convenient to use a redefined set of units based on a redefinition of the energy units. \\

\definition{{\bf The electron-Volt}: 
Equivalent energy of an elementary charge accelerated by an electric field of 1 V/m over a distance of 1m. {\it i.e.} 

$$ 1 eV = 1.6 \cdot 10^{-19} \; {\rm J} $$}

Using the relativistic formulae of rest mass that are detailed in Chapter~\ref{Relativity}, the mass of the proton can be expressed in terms of energy as:

$$ m_p c^2 = 1.66 \cdot 10^{-27} \, {\rm [kg]} \times (3\cdot 10^8)^2 {\rm [m.s^-1]^2} = 1.5 \cdot 10^{-10} \, {\rm J} $$

Therefore 

$$ m_p c^2 = 1.5 \cdot 10^{-10} {\rm [J]} / 1.6 \cdot 10^{-19} \; {\rm [J/eV]} = 0.94 \cdot 10^9 {\rm eV} = 938 MeV$$

The natural unit for Nuclear and lowest to intermediate energy sub-nuclear physics process is the Mega-electron-volts {\bf MeV}.



\section{Natural units}
We have already seen how keeping fundamental constants in calculation
(like $c$, $\hslash$ or $e$) is quite uncomfortable. We have also
expressed energy and masses in electronvolt, which allows us to write
particle masses using simple prefixes in spite of using factors like
$10^{-27}$ to write the mass of a proton in kilograms.

It is common practice, in particle physics but also in other fields,
to use a system of units in which $c = \hslash =1 $. In this way the
velocity is dimensionless, time has the same measure of lengths and
masses and momentum are measured as energies. This system is called
\emph{System of natural units}.

In some cases is also useful to adopt a system in which
$\epsilon_0 = \mu_0 = 1$ (Heavyside--Lorentz units).

In the S.I. we have:
\[ [\hslash] = [M][L^2][T^{-1}]\] then, in natural units:
\[ [M] = [L^{-1}] = [T^{-1}]\] and we are allowed to write:
\begin{eqnarray*}
  1\,\kilo\gram &=& 5.6\,\times\,10^{24}\,\giga\electronvolt\\
  1\,\meter &=& 5.1\,\times\,10^{15}\,\giga\electronvolt^{-1}\\
  1\,\second &=& 1.5\,\times\,10^{24}\,\giga\electronvolt^{-1}\\
\end{eqnarray*}
and also:
\begin{eqnarray*}
  1\,\second &=& 3\,\times\,10^{8}\,\meter\\
  1\,\angstrom &=& \frac{1}{200\,\electronvolt}\\
\end{eqnarray*}

\begin{table}[h!]
\centering
  \begin{tabular}{@{}lllll@{}}
    \toprule
    & \multicolumn{3}{c}{MKS}     & Natural     \\ \midrule
    Quantity & $[M]^p$ & $[L]^q$ & $[T]^r$ & $E^{p-q-r}$ \\ \midrule
    Action   & 1       & 2       & -1      & 0           \\
    Velocity & 0       & 1       & -1      & 0           \\
    Mass     & 1       & 0       & 0       & 1           \\
    Length   & 0       & 1       & 0       & -1          \\
    Time     & 0       & 0       & 1       & -1          \\
    Momentum & 1       & 1       & -1      & 1           \\
    Energy   & 1       & 2       & -2      & 1           \\
    $\alpha$ & 0       & 0       & 0       & 0           \\ \bottomrule
  \end{tabular}
\end{table}

\section{Units of radioactivity}

The activity of a radioactive source is defined as the number of
decays per unit of time. An unit often associated with the activity is
the \emph{curie}:
\[1\,\curie = 3.7\times10^{10} \text{decays per }\second\] and the
Becquerel corresponds to one decay per second, so:
\[1\,\becquerel = 0.27\times 10^{-10} \curie\] Let's consider Radium
as example, a nucleus of $^{226}\text{Ra}$ decays emitting an
alpha--particle with $E = 4.5\,\mega\electronvolt$, with a
$T_{1/2} = 1602 \text{y} = 5.052 \times 10^{10}\,\second$. A gram of
radium is made of
\[\frac{N_A}{A} = \frac{6.022\times 10^{23}}{226} = 2.7\times
  10^{21}\] nuclei, and has a corresponding activity of:
\[\mathcal{A} = \frac{N_A}{A\,\tau} = 3.7\times 10^{10} \,\second^{-1}
  = 1\,\curie\].  For comparison, the uranium $^{235}\text{U}$ has an
half--life of $T_{1/2} = 7.4 \times 10^8 \text{y}$.
