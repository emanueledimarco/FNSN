%%%%%%%%%%%%%%%%%%%%%%preface.tex%%%%%%%%%%%%%%%%%%%%%%%%%%%%%%%%%%%%%%%%%
% sample preface
%
% Use this file as a template for your own input.
%
%%%%%%%%%%%%%%%%%%%%%%%% Springer %%%%%%%%%%%%%%%%%%%%%%%%%%

\preface

These lectures are an introduction to Nuclear and Subnuclear physics at the third year level of the {\it "Laurea Triennale"} in physics. The idea is to try to give an overview of the main concepts that brought to founding new branch of physics at the turn of the XX century. These are a necessary first step into the fundamental concepts of particle and nuclear physics.  \\

Subjects are approximately ordered in time as they have been developed in order to give, as much as possible, a logical order in which the theoretical developments and discoveries have been made. It is particularly interesting to note when theoretical insights and predictions were corroborated by experiment and when instead experimental discoveries have triggered theoretical developments. \\

The lectures are structured in two main parts. The first block of chapters (from the introduction to experimental methods) will set the stage. After an introduction to fundamental aspects in physics (1) and an historic recap from ancient to modern atomism (2), an overview of special relativity, the first fundamental building block of particle and nuclear physics will be discussed (3). Up to this point, these are essentially all recaps from other lectures. Chapter 4 then discusses the second essential building block in understanding nature at the smallest scales {\it i.e.} scattering theory. This will really only be an introduction to scattering theory, based on an extremely important example: the Rutherford scattering experiment. The predictions for the Rutherford scattering will be derived with a classical and a quantum mechanical formalism. This chapter will lead us to an extremely important concept in particle physics: Feynman diagrams and their interpretation in terms of fundamental interactions. This again will only be an introduction. The first part of these lecture will be completed by Chapter 5 on interactions of particles with matter and Chapter 6 on experimental methods. Most of the concepts discussed in this first part can be understood from basic mechanics, electromagnetism and quantum mechanics. \\

The second part of the lectures starts at Chapter 7, when the conclusions of the Rutherford experiment require an explanation of how same sign charges can be concentrated in such a small volume at the center of atom within the nucleus. Chapter 7 will discuss fundamental observations of nuclear radioactivity that are key to understanding the developments of nuclear models and the need for new types of fundamental interactions. Chapter 7 will then attempt to model the new interactions, introducing notions of relativistic quantum mechanics and their implications in terms of the existence of anti-particles. Chapter 8 will discuss the central role played by symmetries in nuclear and particle physics and Chapter 9 discusses models for the nucleus. \\

A few, hopefully most representative, fundamental discoveries, that have lead to the developments of particle and nuclear physics are discussed in Chapter 11. Before concluding a synoptic review of particle properties will be given in Chapter 12.\\

%% Please write your preface here
%Use the template \emph{preface.tex} together with the Springer document class SVMono (monograph-type books) or SVMult (edited books) to style your preface in the Springer layout.

%A preface\index{preface} is a book's preliminary statement, usually written by the \textit{author or editor} of a work, which states its origin, scope, purpose, plan, and intended audience, and which sometimes includes afterthoughts and acknowledgments of assistance. 

%When written by a person other than the author, it is called a foreword. The preface or foreword is distinct from the introduction, which deals with the subject of the work.

%Customarily \textit{acknowledgments} are included as last part of the preface.
 

%\vspace{\baselineskip}
%\begin{flushright}\noindent
%Place(s),\hfill {\it Firstname  Surname}\\
%month year\hfill {\it Firstname  Surname}\\
%\end{flushright}


